\documentclass{beamer}

\usepackage[utf8]{inputenc}
\usepackage[brazil]{babel}
\usepackage{graphicx,hyperref,icmc,url}

\usepackage{pdfpages}
\usepackage{caption}
\usepackage{enumitem}

\usepackage{multirow}
\definecolor{blue(pigment)}{rgb}{0.2, 0.2, 0.6}

% The title of the presentation:
%  - first a short version which is visible at the bottom of each slide;
%  - second the full title shown on the title slide;
\title[Intratabilidade]{Intratabilidade}

% Optional: a subtitle to be dispalyed on the title slide
\subtitle{}

% The author(s) of the presentation:
%  - again first a short version to be displayed at the bottom;
%  - next the full list of authors, which may include contact information;
\author[André, Eduardo, Gustavo, Matheus, Thiago]{
André Luís Mendes Fakhoury\\ Eduardo Dias Pennone\\ Gustavo Vinícius Vieira Silva Soares\\ Matheus Steigenberg Populim\\ Thiago Preischadt Pinheiro\\ \bigskip
\textsc{SCC0205 - Teoria da Computação e Linguagens Formais}\\
Prof. Diego Raphael Amancio
}

% The institute:
%  - to start the name of the university as displayed on the top of each slide
%    this can be adjusted such that you can also create a Dutch version
%  - next the institute information as displayed on the title slide
\institute[ICMC/USP]{ICMC - USP}

% Add a date and possibly the name of the event to the slides
%  - again first a short version to be shown at the bottom of each slide
%  - second the full date and event name for the title slide
\date[2020]{\footnotesize{Novembro de 2020}}

\AtBeginSection[]
{
	\begin{frame}<beamer>{Sumário}
	\tableofcontents[currentsection]
\end{frame}
}

\begin{document}


\begin{frame}[plain]
\titlepage
\end{frame}

\begin{frame}
\frametitle{Sumário}
\tableofcontents
\end{frame}

%%%%%%%%%%%%%%%%%%%%%%%%%%%%%%%%%%%%%%%%%%%%%%%%%%%%%%%%%%%%%%%%
\section{Introdução} %2min

\begin{frame}
\frametitle{Introdução}
\end{frame}

%%%%%%%%%%%%%%%%%%%%%%%%%%%%%%%%%%%%%%%%%%%%%%%%%%%%%%%%%%%%%%%%
\section{Computação eficiente} %2min

\begin{frame}
\frametitle{Computação eficiente}
\framesubtitle{Visão geral}
\end{frame}

%%%%%%%%%%%%%%%%%%%%%%%%%%%%%%%%%%%%%%%%%%%%%%%%%%%%%%%%%%%%%%%%
\section{Classes P e NP}

\begin{frame}
\frametitle{Classes P e NP}
\framesubtitle{Visão geral}
\end{frame}

%%%%%%%%%%%%%%%%%%%%%%%%%%%%%%%%%%%%%%%%%%%%%%%%%%%%%%%%%%%%%%%%
\section{Hipóteses P $\neq$ NP}

\begin{frame}
\frametitle{Hipóteses P $\neq$ NP}
\framesubtitle{Visão geral}
\end{frame}

%%%%%%%%%%%%%%%%%%%%%%%%%%%%%%%%%%%%%%%%%%%%%%%%%%%%%%%%%%%%%%%%
\section{Reduções em tempo polinomial}

\begin{frame}
\frametitle{Reduções em tempo polinomial}
\framesubtitle{Visão geral}
\end{frame}

%%%%%%%%%%%%%%%%%%%%%%%%%%%%%%%%%%%%%%%%%%%%%%%%%%%%%%%%%%%%%%%%
\section{NP-difícil}

\begin{frame}
\frametitle{NP-difícil}
\framesubtitle{Visão geral}
\end{frame}

%%%%%%%%%%%%%%%%%%%%%%%%%%%%%%%%%%%%%%%%%%%%%%%%%%%%%%%%%%%%%%%%
\section{NP-completo}

\begin{frame}
\frametitle{NP-completo}
\framesubtitle{Introdução}

\begin{itemize}
    \item $\bullet$ Década de 1970 - Stephen Cook e Leonid Levin
    \item $\bullet$ Alguns problemas em NP possuem complexidade individual relacionada com toda a classe
    \item $\bullet$ Caso um algoritmo em tempo polinomial exista para algum deles, todos os problemas em NP podem ser resolvidos em tempo polinomial
\end{itemize}
\bigskip

\end{frame}

\begin{frame}
\frametitle{NP-completo}
\framesubtitle{Definição}


\begin{block}{Definição}
Uma linguagem $B$ é dita \textit{NP-completa} caso satisfaça estas duas características:
\begin{itemize}
	\item 1. $B$ está em $NP$;
	\item 2. Todo $A$ em $NP$ é redutível em tempo polinomial para $B$.
\end{itemize}
\end{block}
\end{frame}

\begin{frame}
\frametitle{NP-completo}
\framesubtitle{Definição}

Também pode-se definir a partir da classe NP-difícil:

\begin{block}{Definição}
	Uma linguagem $B$ é dita \textit{NP-completa} caso satisfaça estas duas características:
	\begin{itemize}
		\item 1. $B$ está em $NP$;
		\item 2. $B$ é NP-difícil.
	\end{itemize}
\end{block}

\end{frame}

\begin{frame}
\frametitle{NP-completo}
\framesubtitle{Definição}

\begin{block}{Teorema}
	Se $B$ é \textit{NP-completo} e $B \leq_p C$ ($B$ é redutível em tempo polinomial para $C$), para $C$ em $NP$, então $C$ é \textit{NP-completo}.
\end{block}

Com isso, pode-se utilizar um problema NP-completo para encontrar outros problemas NP-completos!

\end{frame}

\begin{frame}
\frametitle{NP-completo}
\framesubtitle{Consequências}

\begin{itemize}
    \item $\bullet$ Novos problemas NP-completos podem ser encontrados a partir de problemas NP-completos já conhecidos
    \item $\bullet$ Achar uma solução polinomial para algum NP-completo implica que todos os NP possuem solução polinomial
    \item $\bullet$ Estudos podem ser mais focados diretamente nos NP-completos
\end{itemize}

\bigskip

\end{frame}

\begin{frame}
\frametitle{NP-completo}
\framesubtitle{Problemas}

\begin{itemize}
    \item $\bullet$ \textbf{Teorema de Cook-Levin: }SAT
    \item $\bullet$ 21 problemas de Karp - \textit{Reducibility among combinatorial problems} \cite{karp1972}
    \item $\bullet$ SAT, 3SAT, clique, caixeiro viajante (decisão), ...
\end{itemize}

\bigskip

\end{frame}

\begin{frame}
\frametitle{NP-completo}
\framesubtitle{SAT (Satisfatibilidade booleana)}

\textcolor{blue(pigment)}{\textbf{SAT}}

Verificar se existe uma valoração para as variáveis que compõe uma fórmula booleana, de forma que esta tenha saída verdadeira.

$$\text{Exemplo: }(x \lor y \lor z) \wedge (\neg x \lor y) \wedge (y \lor \neg z)$$

Neste caso, pode-se atribuir $x, y, z$ como \textit{verdadeiros}, a saída será \textit{verdadeira}.

\end{frame}

\begin{frame}
\frametitle{NP-completo}
\framesubtitle{SAT (Satisfatibilidade booleana)}

\begin{itemize}
    \item $\bullet$ Primeiro problema provado NP-completo, pelo \textbf{Teorema de Cook-Levin};
    \item $\bullet$ Auto-redutível: algoritmo que resolve o problema (e verifica se existe tal composição de valores para variáveis) também consegue encontrar a atribuição de variáveis satisfatórias;
\end{itemize}
\end{frame}

\begin{frame}
\frametitle{NP-completo}
\framesubtitle{3SAT}

\textcolor{blue(pigment)}{\textbf{3SAT}}

SAT com expressões na forma normal conjuntiva, e cada cláusula contém exatamente três variáveis.

$$(x \lor y \lor z) \wedge (\neg x \lor \neg w \lor z) \wedge (x \lor w \lor \neg z)$$  

\end{frame}

\begin{frame}
\frametitle{NP-completo}
\framesubtitle{3SAT}

\textcolor{blue}{\textbf{Ideia da demonstração}}

É \textbf{NP} pois dada uma coleção de cláusulas e uma valoração para as variáveis, é possível verificar em tempo polinomial se estas cláusulas são satisfeitas.

\bigskip

$\text{SAT} \leq_p \text{3SAT}$ - expressões são denotadas na forma normal conjuntiva, e cláusulas com mais de três termos são substituídas por mais cláusulas com três termos.

\end{frame}

\begin{frame}
\frametitle{NP-completo}
\framesubtitle{Clique}
\textcolor{blue(pigment)}{\textbf{Clique}}

Dado um grafo $G(V, A)$ e um inteiro $k$, encontrar se existe um subgrafo de $k$ ou mais vértices, em que todos nós dois a dois são conectados por uma arestas - ou seja, se $G$ possui um conjunto de $k$ nós mutualmente adjacentes.

\end{frame}

\begin{frame}
\frametitle{NP-completo}
\framesubtitle{Clique}

\textcolor{blue}{\textbf{Ideia da demonstração}}

É \textbf{NP} pois é possível construir um verificador que recebe um grafo $G(V, A)$, um inteiro $k$ e um conjunto $S$, e verifica se $S$ é um clique válido em $\mathcal{O}(|S|^2)$.

\bigskip

$\text{3SAT} \leq_p \text{CLIQUE}$ - constrói-se um grafo a partir da expressão booleana. Será feita a construção de uma função que recebe qualquer instância do 3SAT e retorna uma instância de Clique, que é verdadeira se e somente se a instância do 3SAT for verdadeira.

\end{frame}


\begin{frame}
\frametitle{NP-completo}
\framesubtitle{Cobertura de vértices (decisão)}

\textcolor{blue(pigment)}{\textbf{Cobertura de vértices (decisão)}}

Dado um grafo $G(V, A)$ e um inteiro $k$, encontrar se existe um subconjunto de vértices $V'$ de tamanho $k$ tal que toda aresta em $A$ esteja conectada em no mínimo um vértice em $V'$.

\end{frame}

\begin{frame}
\frametitle{NP-completo}
\framesubtitle{Cobertura de vértices (decisão)}

\textcolor{blue}{\textbf{Ideia da demonstração}}

\textbf{NP: } dado um subconjunto $V'$, pode-se verificar com complexidade polinomial se este é um subconjunto correto (basta percorrer por todas as arestas e verificar se algum de seus vértices pertence ao conjunto $V'$).

\bigskip

$\text{CLIQUE} \leq_p \text{Vertex Cover}$: dado um grafo $G(V, A)$, seu complemento $\bar G(V, \bar A)$ possui os mesmos vértices de $G$, e nenhuma de suas arestas (mas sim todas as arestas que não estão em $G$). Se existe um clique $V'$ em $\bar G$, então $V - V'$ é uma cobertura de vértices em $G$.

\end{frame}

\begin{frame}
\frametitle{NP-completo}
\framesubtitle{Caminho hamiltoniano}

\textcolor{blue(pigment)}{\textbf{Caminho hamiltoniano (HAMPATH)}}

O problema do caminho hamiltoniano consiste em, dado um grafo $G(V, A)$ e dois vértices $s, t \in V$, verificar se existe um caminho de $s$ até $t$ que passa por todos os vértices $\in V$ exatamente uma vez.

\end{frame}

\begin{frame}
\frametitle{NP-completo}
\framesubtitle{Caminho hamiltoniano}

\textcolor{blue}{\textbf{Ideia da demonstração}}

\textbf{NP: } Dado um caminho $C$, pode-se facilmente verificar se este é um caminho válido com complexidade polinomial $\mathcal{O}(|V| + |A|)$.

\bigskip

É possível mostrar que $\text{3SAT} \leq_p \text{HAMPATH}$.

\end{frame}

\begin{frame}
\frametitle{NP-completo}
\framesubtitle{Caixeiro viajante (decisão)}

\textcolor{blue(pigment)}{\textbf{Caixeiro viajante (decisão)}}

Dados um grafo $G(V, A)$ e uma distância $L$, retornar se existe um ciclo (que visite todos os vértices exatamente uma vez) com distância total no máximo $L$.
\end{frame}

\begin{frame}
\frametitle{NP-completo}
\framesubtitle{Caixeiro viajante (decisão)}

\textcolor{blue}{\textbf{Ideia da demonstração}}

Dado um grafo $G(V, A)$, um inteiro $L$ e um ciclo $c$, pode-se verificar se o ciclo $c$ de fato é um ciclo válido neste grafo e se a soma de suas arestas não ultrapassa $L$, com complexidade $\mathcal{O}(|c|)$.

\bigskip

É possível mostrar que $\text{HamCycle} \leq_p \text{TSP}$.

\end{frame}

% \begin{figure}[hbt]
%     \begin{center}
%     \caption{Análise em um labirinto aleatório (735x304)}
%     \includegraphics[width=0.8\textwidth]{graphs/maze1.png}
%     \end{center}
% \end{figure}

%%%%%%%%%%%%%%%%%%%%%%%%%%%%%%%%%%%%%%%%%%%%%%%%%%%%%%%%%%%%%%%%
\section{Conclusão}

\begin{frame}
\frametitle{Conclusão}

\begin{itemize}
	\item $\bullet$ Área muito importante e interessante.
    \item $\bullet$ Classificação de problemas em classes;
    \item $\bullet$ Provar que determinados problemas são "tão difíceis" quanto vários outros estudados;
    \item $\bullet$ Não ``perder tempo'' com problemas já considerados difíceis;
    \item $\bullet$ Espera-se que $P \neq NP$, mas ainda não foi provado;
\end{itemize}

\end{frame}

%%%%%%%%%%%%%%%%%%%%%%%%%%%%%%%%%%%%%%%%%%%%%%%%%%%%%%%%%%%%%%%%
\section{Referências}

\nocite{*}

\begin{frame}[allowframebreaks]
\frametitle{Referência Bibliográfica}

\bibliographystyle{plain} % We choose the "plain" reference style
\bibliography{refs} % Entries are in the "refs.bib" file

\end{frame}

\end{document}
