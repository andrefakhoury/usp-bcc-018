\section{Introdução}

\begin{frame}{Introdução}
\framesubtitle{Histórico}
\begin{itemize}
    \item O conceito de programação orientada a objetos teve suas raízes no SIMULA 67, mas não foi totalmente desenvolvido até que a evolução do Smalltalk resultou no Smalltalk 80. 
    \item Muitos consideram o Smalltalk como sendo o modelo básico para uma linguagem de programação puramente orientada a objetos.
\end{itemize} 
\end{frame}

\begin{frame}{Introdução}
\framesubtitle{O paradigma da orientação a objetos}
\begin{itemize}
    \item Uma linguagem orientada a objetos deve fornecer suporte para três recursos-chave da linguagem: tipos de dados abstratos, herança e vinculação dinâmica de chamadas de método a métodos.
    \item Com base nesses recursos temos os 4 pilares da orientação a objetos: Abstração, Encapsulamento, Herança e Polimorfismo.
\end{itemize} 
\end{frame}

\section{Conceitos importantes presentes no paradigma}

\begin{frame}{Conceitos importantes presentes no paradigma}
\framesubtitle{Classe}
\begin{itemize}
    \item Uma abstração de um conjunto de objetos que possuem características em comum.
    \item A descrição de propriedades, estados, comportamentos e ações de um conjunto de objetos.
\end{itemize} 
\end{frame}

\begin{frame}{Conceitos importantes presentes no paradigma}
\framesubtitle{Objeto}
\begin{itemize}
    \item A instância de uma classe.
    \item Representação de um objeto do mundo real pertencente a uma classe especificada.
    \item Caracterizado por atributos e métodos.
\end{itemize} 
\end{frame}

\begin{frame}{Conceitos importantes presentes no paradigma}
\framesubtitle{Atributos}
\begin{itemize}
    \item As características, propriedades ou estados de um objeto.
    \item Exemplos: tamanho, cor, peso. 
\end{itemize} 
\end{frame}

\begin{frame}{Conceitos importantes presentes no paradigma}
\framesubtitle{Métodos}
\begin{itemize}
    \item Os comportamentos, ações ou funcionalidades que um objeto pode desempenhar.
    \item Exemplos: comer, voar, latir.
\end{itemize} 
\end{frame}

\begin{frame}{Conceitos importantes presentes no paradigma}
\framesubtitle{Abstração}
\begin{itemize}
    \item Uma abstração é uma visão ou representação de uma entidade que inclui apenas os atributos mais significativos. Em um sentido geral, a abstração permite coletar instâncias de entidades em grupos nos quais seus atributos comuns não precisam ser considerados.
\end{itemize} 
\end{frame}

\begin{frame}{Conceitos importantes presentes no paradigma}
\framesubtitle{Herança}
\begin{itemize}
    \item A herança possibilita que as classes compartilhem seus métodos e atributos entre si seguindo uma hierarquia baseada nos níveis de abstração.
    \item Facilitam a reutilização do código, uma vez que atributos e métodos das classes superiores podem ser aproveitados.
\end{itemize} 
\end{frame}

\begin{frame}{Conceitos importantes presentes no paradigma}
\framesubtitle{Encapsulamento}
\begin{itemize}
    \item Controla o acesso à atributos e métodos de uma classe.
    \item Evitar o acesso direto a propriedade do objeto adiciona segurança à aplicação.
    \item O código pode evoluir a partir de uma interface em comum.
\end{itemize} 
\end{frame}

\begin{frame}{Conceitos importantes presentes no paradigma}
\framesubtitle{Polimorfismo}
\begin{itemize}
    \item Consiste na alteração do funcionamento interno de um método herdado de um objeto pai.
    \item Subclasses diferentes que possuem uma mesma classe pai, possuem comportamentos diferentes em relação ao mesmo método herdado.
\end{itemize} 
\end{frame}