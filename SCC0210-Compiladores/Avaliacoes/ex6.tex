\documentclass{article}
\usepackage[a4paper, total={7in, 9in}]{geometry}
\usepackage[brazil,portuges]{babel}
\usepackage[utf8]{inputenc}
\usepackage[T1]{fontenc}
\usepackage{ebgaramond}
\usepackage{amsmath}
\usepackage{indentfirst}
\usepackage{titlesec}
\usepackage{todonotes}
\usepackage{enumitem}
\usepackage{tikz}
\usepackage{array}
\usepackage{listings}% http://ctan.org/pkg/listings
\lstset{
  basicstyle=\ttfamily,
  mathescape
}

\title{Compiladores - Exercício 6}
\author{André L. Mendes Fakhoury\\
Gustavo V. V. Silva Soares\\
Eduardo Dias Pennone\\
Matheus S. Populim\\
Thiago Preischadt\\
}
\date{2021}

\begin{document}

\maketitle

\section{Construa a tabela sintática para a gramática abaixo e reconheça a cadeia $id+id*id$ utilizando análise sintática preditiva não recursiva.}

\begin{lstlisting}
<E>  ::= <T><E'>
<E'> ::= +<T><E'> | $\lambda$
<T>  ::= <F><T'>
<T'> ::= *<F><T'> | $\lambda$
<F>  ::= (E) | id
\end{lstlisting}

Primeiramente, devemos calcular o conjunto primeiro e seguidor de cada símbolo não terminal. Começando com o conjunto primeiro:

\begin{lstlisting} 
P(E') = {+, $\lambda$}
P(T') = {*, $\lambda$}
P(F)  = {(, id}
P(T)  = P(F) = {(, id}
P(E)  = P(T) = {(, id}
\end{lstlisting}

Calculando o seguidor:

\begin{lstlisting} 
S(E)  = {), $\lambda$}
S(E') = S(E) = {), $\lambda$}
S(T)  = P(E') U S(E) U S(E') = {+, ), $\lambda$}
S(T') = S(T) = {+, ), $\lambda$}
S(F)  = P(T') U S(T) U S(T') = {*, +, $\lambda$}
\end{lstlisting}

A partir destes conjuntos, podemos construir a tabela sintática. Iniciando o processo a partir de informações do conjunto primeiro, temos:

\begin{center}
\begin{tabular}{ |c|m{2cm}|m{2cm}|m{2cm}|m{2cm}|m{2cm}|m{2cm}| } 
 \hline
 & $id$ & $+$ & $*$ & $($ & $)$ & $\lambda$ \\
 \hline
 $E$ & $E \rightarrow TE'$ & $ $ & $ $ & $E \rightarrow TE'$ & $ $ & $ $ \\
 \hline
 $E'$ & $ $ & $E' \rightarrow +TE'$ & $ $ & $ $ & $ $ & $E' \rightarrow \lambda$ \\
 \hline
 $T$ & $T \rightarrow FT'$ & $ $ & $ $ & $T \rightarrow FT'$ & $ $ & $ $ \\
 \hline
 $T'$ & $ $ & $ $ & $T' \rightarrow *FT'$ & $ $ & $ $ & $T' \rightarrow \lambda$ \\
 \hline
 $F$ & $F \rightarrow id$ & $ $ & $ $ & $F \rightarrow (E)$ & $ $ & $ $ \\
 \hline
\end{tabular}
\end{center}

Porém, também temos que analisar os casos em que temos $A \rightarrow \alpha$, e $P(\alpha)$ contém $\lambda$. Fazendo isso, temos a tabela sintática:

\begin{center}
\begin{tabular}{ |c|m{2cm}|m{2cm}|m{2cm}|m{2cm}|m{2cm}|m{2cm}| } 
 \hline
 & $id$ & $+$ & $*$ & $($ & $)$ & $\lambda$ \\
 \hline
 $E$ & $E \rightarrow TE'$ & $ $ & $ $ & $E \rightarrow TE'$ & $ $ & $ $ \\
 \hline
 $E'$ & $ $ & $E' \rightarrow +TE'$ & $ $ & $ $ & $E' \rightarrow \lambda$ & $E' \rightarrow \lambda$ \\
 \hline
 $T$ & $T \rightarrow FT'$ & $ $ & $ $ & $T \rightarrow FT'$ & $ $ & $ $ \\
 \hline
 $T'$ & $ $ & $T' \rightarrow \lambda$ & $T' \rightarrow *FT'$ & $ $ & $T' \rightarrow \lambda$ & $T' \rightarrow \lambda$ \\
 \hline
 $F$ & $F \rightarrow id$ & $ $ & $ $ & $F \rightarrow (E)$ & $ $ & $ $ \\
 \hline
\end{tabular}
\end{center}

As células vazias indicam erro.

A partir da tabela sintática, podemos reconhecer (ou não) a cadeia $id + id * id$. O passo a passo é o seguinte:

\begin{center}
\begin{tabular}{c|c|c}
Pilha & Cadeia & Regra\\
\hline
$\lambda E$ & $id + id * id\lambda$ & $E \rightarrow TE'$\\
$\lambda E'T$ & $id + id * id\lambda$ & $T \rightarrow FT'$\\
$\lambda E'T'F$ & $id + id * id\lambda$ & $F \rightarrow id'$\\
$\lambda E'T'id$ & $id + id * id\lambda$ & $\cdots$\\
$\lambda E'T'$ & $+ id * id\lambda$ & $T' \rightarrow \lambda$\\
$\lambda E'$ & $+ id * id\lambda$ & $E' \rightarrow +TE'$\\
$\lambda E'T+$ & $+ id * id\lambda$ & $\cdots$\\
$\lambda E'T$ & $id * id\lambda$ & $T \rightarrow FT'$\\
$\lambda E'T'F$ & $id * id\lambda$ & $F \rightarrow id$\\
$\lambda E'T'id$ & $id * id\lambda$ & $\cdots$\\
$\lambda E'T'$ & $* id\lambda$ & $T' \rightarrow *FT'$\\
$\lambda E'T'F*$ & $* id\lambda$ & $\cdots$\\
$\lambda E'T'F$ & $id\lambda$ & $F \rightarrow id$\\
$\lambda E'T'id$ & $id\lambda$ & $\cdots$\\
$\lambda E'T'$ & $\lambda$ & $T' \rightarrow \lambda$\\
$\lambda E'$ & $\lambda$ & $E' \rightarrow \lambda$\\
$\lambda$ & $\lambda$ & $SUCESSO$\\
\end{tabular}
\end{center}

Assim, reconhecemos a cadeia citada anteriormente utilizando análise sintática preditiva não recursiva.

\end{document}
