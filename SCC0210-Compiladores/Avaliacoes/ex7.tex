\documentclass{article}
\usepackage[a4paper, total={7in, 9in}]{geometry}
\usepackage[brazil,portuges]{babel}
\usepackage[utf8]{inputenc}
\usepackage[T1]{fontenc}
\usepackage{ebgaramond}
\usepackage{amsmath}
\usepackage{indentfirst}
\usepackage{titlesec}
\usepackage{todonotes}
\usepackage{enumitem}
\usepackage{tikz}
\usepackage{array}
\usepackage{listings}% http://ctan.org/pkg/listings
\lstset{
  basicstyle=\ttfamily,
  mathescape
}
% oi abiguinhos %
\title{Compiladores - Exercício 7}
\author{André L. Mendes Fakhoury\\
Gustavo V. V. Silva Soares\\
Eduardo Dias Pennone\\
Matheus S. Populim\\
Thiago Preischadt\\
}
\date{2021}

\begin{document}

\maketitle

\section{Método mecânico}

Construir  a  tabela  sintática  para  a  gramática  abaixo  pelo  método mecânico e reconhecer cadeia $(a*b)$

\begin{lstlisting}
S ::= ( S O S ) | a | b
O ::= + | *
\end{lstlisting}

A gramática não é de precedência de operadores, pois há símbolos não terminais adjacentes. Transformando-a em uma gramática de precedência de operadores:
\begin{lstlisting}
S ::= ( S + S ) | ( S * S ) | a | b
\end{lstlisting}

Os primeiros terminais de $S$ são $\{(, a, b\}$ e os últimos são $\{), a, b\}$

A partir dos pares do tipo $aX$, que são $( S$, $+ S$ e $* S$, podemos extrair as seguintes relações: $\{ (, +, * \} < \{ (, a, b \}$.

A partir dos pares do tipo $Xb$, que são $S +$, $S )$ e $S *$, podemos extrair as seguintes relações: $\{ ), a, b \} > \{ +, ), * \}$.

A partir das sequências do tipo $a\beta b$, que são $( S +$, $+ S )$, $( S *$ e $* S )$, podemos extrair as seguintes relações: $( = +$, $+ = )$, $( = *$ e $* = )$.

Para $\$$, tem-se as relações: $\$ < \{ (, a, b \}$ e $\{ ), a, b \} > \$$.

A tabela sintática é:
\begin{center}
\begin{tabular}{ |c|c|c|c|c|c|c|c| } 
\hline
  & + & * & ( & ) & a & b & \$ \\
\hline
+ &   &   & < & = & < & < &  \\
\hline
* &   &   & < & = & < & < &  \\
\hline
( & = & = & < &   & < & < &  \\
\hline
) & > & > &   & > &   &   & >\\
\hline
a & > & > &   & > &   &   & >\\
\hline
b & > & > &   & > &   &   & >\\
\hline
\$ &   &   & < &   & < & < &  \\
\hline
\end{tabular}
\end{center}

O reconhecimento da cadeia $(a*b)$ é feito da seguinte forma:
\begin{center}
\begin{tabular}{ |c|c|c| } 
\hline
pilha & cadeia & ação\\
\hline
\$< & (a*b)\$ & empilha\\
\hline
\$<(< & a*b)\$ & empilha\\
\hline
\$<(<a> & *b)\$ & reduz\\
\hline
\$<(= & *b)\$ & empilha\\
\hline
\$<(=*< & b)\$ & empilha\\
\hline
\$<(=*<b> & )\$ & reduz\\
\hline
\$<(=*= & )\$ & empilha\\
\hline
\$<(=*=)> & \$ & reduz\\
\hline
\$ & \$ & aceita\\
\hline
\end{tabular}
\end{center}

\section{Método intuitivo}

Construir a tabela sintática para a gramática abaixo:

\begin{lstlisting}
<E> ::= <E>+<E> | <E>*<E> | <E>**<E> | (<E>) | id
\end{lstlisting}

Como sabemos que a ordem das precedências, em ordem decrescente, é: \texttt{**} (que é associativo a direita), \texttt{*} (que é associativo à esquerda); \texttt{+} (que também é associativo à esquerda), podemos gerar a tabela sintática:

\begin{itemize}
    \item Pela ordem das precedências dos operadores: \texttt{**>*, *<**, **>+, +<**, *>+, +<*}
    \item Pela associatividade dos operadores: \texttt{*>*, +>+, **<**}
    \item As outras relações são fixas: operador x tem x>\$, \$<x, x<id, id>x, x<(, (<x, x>), )>x; entre operandos, (<(, )>), id>), \$<(, (=), )>\$, id>\$, \$<id, (<id
\end{itemize}

Com estes itens, podemos gerar a seguinte tabela sintática:

\begin{center}
\begin{tabular}{ |c|c|c|c|c|c|c|c| } 
\hline
& id & ** & * & + & ( & ) & \$ \\
\hline
id &  & > & > & > &  & > & > \\
\hline
** & <  & <  & > & > & < & > & > \\
\hline
* & < & < & > & > & < & > & > \\
\hline
+ & < & < & < & > & < & > & > \\
\hline
( & < & < & < & < & < & = & \\
\hline
) &   & > & > & > &   & > & > \\
\hline
\$ & < & < & < & < & < &   &   \\
\hline

 
\end{tabular}
\end{center}


\end{document}
