\documentclass{article}
\usepackage[utf8]{inputenc}
\usepackage[T1]{fontenc}
\usepackage{ebgaramond}
\usepackage{amsmath}
\usepackage{indentfirst}
\usepackage{titlesec}
\usepackage{enumitem}
\usepackage{tikz}

\title{Compiladores - Exercício 1}
\author{André L. Mendes Fakhoury\\
Gustavo V. V. Silva Soares\\
Eduardo Dias Pennone\\
Matheus S. Populim\\
Thiago Preischadt\\
}
\date{2021}

\begin{document}

\maketitle

\section{Quais as características de uma linguagem que determinam que ela deve ser compilada ou interpretada? %Atenção: a questão refere-se à linguagem em si, independentemente do uso que é feito dela
}

A princípio, uma linguagem pode ser implementada tanto como compilada quanto interpretada, ou até mesmo híbrida, a depender primordialmente do uso a que se quer atingir (linguagens interpretadas precisam executar trechos do código fonte diversas vezes e portanto não tem como prioridade tanta velocidade, enquanto linguagens compiladas tem como foco realizar a compilação apenas uma vez para que a execução seja repetida mais de uma vez). Porém, costumeiramente, as linguagens podem ter algumas características que guiam se ela será compilada ou interpretada. Por exemplo, o comando ``eval'' (que executa instruções do código fonte durante a interpretação) normalmente é utilizado em linguagens interpretadas (como \textit{javascript} e \textit{python}), por praticidades de implementação.

\section{Por que estudar compiladores? %(outras razões)?
}

O entendimento de como uma linguagem é transformada em código de máquina permite que possamos tomar escolhas conscientes ao longo do processo de programação, evitando tomar decisões que possam comprometer o desempenho de um programa, e planejando o código de forma que aproveite ao máximo o processo de otimização do compilador.

\end{document}
