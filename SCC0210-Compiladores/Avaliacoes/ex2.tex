\documentclass{article}
\usepackage[brazil,portuges]{babel}
\usepackage[utf8]{inputenc}
\usepackage[T1]{fontenc}
\usepackage{ebgaramond}
\usepackage{amsmath}
\usepackage{indentfirst}
\usepackage{titlesec}
\usepackage{todonotes}
\usepackage{enumitem}
\usepackage{tikz}

\title{Compiladores - Exercício 2}
\author{André L. Mendes Fakhoury\\
Gustavo V. V. Silva Soares\\
Eduardo Dias Pennone\\
Matheus S. Populim\\
Thiago Preischadt\\
}
\date{2021}

\begin{document}

\maketitle

\section{Caminho inverso: definir a gramática das seguintes linguagens}

Podemos definir uma gramática G como uma quádrupla $(V_n, V_t, P, S)$, em que $V_n$ é o conjunto dos símbolos não terminais, $V_t$ é o conjunto dos símbolos terminais, $P$ são as regras de produção e $S$ o símbolo inicial.

\subsection{$L(G): \{a^mb^n\ |\ m, n \geq 1\}$}

$$G = (\{S, B\}, \{a, b\}, P, S)$$
$P = \{$
\begin{align*}
S &\rightarrow aS\\
S &\rightarrow aB\\
B &\rightarrow bB\\
B &\rightarrow b
\end{align*}
$\}$. No caso, $G$ é uma gramática regular.

\subsection{$L(G): \{a^nb^n\ |\ n \geq 1\}$}


$$G = (\{S\}, \{a, b\}, P, S)$$
$P = \{$
\begin{align*}
& S \rightarrow aSb\ |\ ab
\end{align*}
$\}$. No caso, $G$ é uma gramática livre de contexto.

\section{Escreva uma gramática que reconheça as expressões aritméticas (considere números e não variáveis). É possível identificar que a gramática construída é ambígua?}
$$G = (\{E, S, N, C, P\}, \{0, 1, 2, 3, 4, 5, 6, 7, 8, 9, +, -, /, *, (, )\}, P, E)$$
$P = \{$
\begin{align*}
 E &\rightarrow N\\
 E &\rightarrow (ESE)\\
 E &\rightarrow ESE\\
 N &\rightarrow PC\\
 N &\rightarrow 0\ | \ 1\ | \ 2\ | \ 3\ | \ 4\ | \ 5\ | \ 6\ | \ 7\ | \ 8\ | \ 9\\
 P &\rightarrow 1\ | \ 2\ | \ 3\ | \ 4\ | \ 5\ | \ 6\ | \ 7\ | \ 8\ | \ 9\\
 C &\rightarrow CC\\
 C &\rightarrow 0\ | \ 1\ | \ 2\ | \ 3\ | \ 4\ | \ 5\ | \ 6\ | \ 7\ | \ 8\ | \ 9\\
 S &\rightarrow +\ | \ -\ | \ /\ | \ *
\end{align*}
$\}$.\\*
A gramática é ambígua pois existem múltiplas formas de produzir a sequencia $(10 + 1)$.\\
$E \rightarrow (ESE) \rightarrow (NSE) \rightarrow (PCSE) \rightarrow (1CSE) \rightarrow (10SE) \rightarrow (10+E) \rightarrow (10+N) \rightarrow (10+1)$\\
$E \rightarrow (ESE) \rightarrow (ESN) \rightarrow (ES1) \rightarrow (NS1) \rightarrow (PCS1) \rightarrow (P0S1) \rightarrow (P0+1) \rightarrow (10+1)$


\newpage
\section{Classifique a gramática abaixo e identifique sua linguagem}


\subsection{Gramática 1}
$G= ( \{ S,A,B,C,D,E \} , \{ a \} , P,S )$\\
$ P = \{ $
\begin{enumerate}
    \item $S \rightarrow ACaB$
    \item $Ca \rightarrow aaC $
    \item $CB \rightarrow DB$
    \item $CB \rightarrow E$
    \item $aD \rightarrow Da$
    \item $AD \rightarrow AC$
    \item $aE \rightarrow Ea$
    \item $AE \rightarrow \lambda$
\end{enumerate}
$\}$\\
% não entendi muito bem esse exercício
% mas vou dar uma olhada melhor nos vídeos


$
S \rightarrow
ACaB \rightarrow
AaaCB \rightarrow
AaaE \rightarrow
AaEa \rightarrow
AEaa \rightarrow
\lambda aa \rightarrow 
aa \rightarrow a^2
$
\\

$
S \rightarrow
ACaB \rightarrow
AaaCB \rightarrow
AaaDB \rightarrow
AaDaB \rightarrow
ADaaB \rightarrow
ACaaB \rightarrow
AaaCaB \rightarrow
AaaaaCB \rightarrow ...
$
\\

$
4) AaaaaCB \rightarrow 
AaaaaE \rightarrow 
... \rightarrow 
AEaaaa \rightarrow 
\lambda aaaa \rightarrow 
a^4
$
\\

$
3)
AaaaaDB \rightarrow 
... \rightarrow 
ADaaaaB \rightarrow 
ACaaaaB \rightarrow 
AaaCaaaB \rightarrow 
AaaaaCaaB \rightarrow 
AaaaaaaCaB \rightarrow
AaaaaaaaaCB \rightarrow 
... \rightarrow 
AEaaaaaaaa \rightarrow 
a^8
$\\
\\

Como essa gramática possui a regra $CB \rightarrow E$, em que a substituição reduz o comprimento da forma sentencial, ela é definida como \textbf{com estrutura de frase ou irrestrita} e possui linguagem definida formalmente por $L(G):\{a^{2^n}$| $n\geq 1\}$\\
\\


\newpage

\subsection{Gramática 2}
$G= ( {S,A,B}, {a}, P,S )$\\
$ P = \{ $
\begin{enumerate}
    \item $S \rightarrow aB | bA$
    \item $A \rightarrow a | aS | bAA$
    \item $B \rightarrow b | bS | aBB$
\end{enumerate}
$\}$\\

Exemplos de cadeias aceitas por essa linguagem:\\
$
S \rightarrow aB \rightarrow ab\\
S \rightarrow bA \rightarrow ba\\
S \rightarrow aB \rightarrow aaBB \rightarrow aabb\\
S \rightarrow bA \rightarrow bbAA \rightarrow bbaa\\
S \rightarrow aB \rightarrow abS \rightarrow abbA \rightarrow abba\\
S \rightarrow aB \rightarrow abS \rightarrow abaB \rightarrow abab\\
S \rightarrow bA \rightarrow baS \rightarrow baaB \rightarrow baab\\
$
\\

Ou seja, essa linguagem apresenta um número igual e não nulo de símbolos $a$ e $b$. A gramática pode ser classificada como uma \textbf{gramática livre de contexto}, e a linguagem formalmente definida por $L(G): \{ \{a,b\}^{*} , |a| = |b| = n \geq 1 \}$

\end{document}
